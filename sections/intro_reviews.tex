% !TEX root = ../main.tex

\citet{2014PASA...31...35D} introduction:

\begin{itemize}
  \item spirals patterns are present in 2/3 of all massive galaxies (GZ2)
  \item Sites of a majority of a star formation
  \item understanding spirals is essential for understanding star formation and galaxy evolution
  \item hubble classification of spirals are divided by tightness and the presence of a bar
  \item can also include information on bulge size and luminosity, and galaxy gas content
  \item Elmegreen \& Elmegreen proposed another scheme, classifying spirals into 12 types depending on the number and lenght of spiral arms
  \item This was simplified into \textit{flocculent}, \textit{grand design} and \textit{multi-armed} spirals
  \item 60\% of galaxies exhibit some grand design structure
  \item the type of spiral is linked to the mechanism which generated them
  \begin{itemize}
    \item (quasi-stationary) density wave theory (\textit{QSDW})
    \item local instabilities, perturbations or noise which are swing-amplified
    \item tidal interactions
    \item bars may also play a role in inducing spiral arms
  \end{itemize}
  \item \textbf{these mechanisms are not mutually exclusive}
  \item flocculent and multi-arm spirals generally thought to arise from local instabilities
  \item grand designs are thought to have undergone a tidal interaction, have a bar driving arms or be obeying \textit{QSDW}
  \item we can measure spiral number, pitch angle, amplitude, arm shape and lifetime
\end{itemize}

\citet{2019MNRAS.487.1808M} introduction:

\begin{itemize}
  \item Hubble classification is a common technique
  \item Hubble spirals are ordered in a sequence extending away from the ellipticals, and separated depending on the presence of a bar
  \item Extended by de Vaucouleurs to include Sd
  \begin{itemize}
    \item split by spiral arm appearance (how tightly wound and how distinct the arms were)
    \item the prominence of a central bulge
  \end{itemize}
  \item galaxy morphology encodes information on its dynamical history, including its formation and evolution
  \item morphology is known to correlate well with other physical properties
  \item people have taken to using proxies for morphology
  \item this is not a valid approach
  \item this paper explores an updated view of the Hubble sequence with the morphological classifications provided by Willett et al. (2013)
  \item most experts say that classifying based on bulge size vs spiral tightness result in consistent classification
  \item modern automatic galaxy classification conflates bulge size alone with spiral type
  \item differences in arm characteristics is linked to different formation mechanisms (floculent to shearing, grand design to density waves)
\end{itemize}

\citet{2019arXiv190804246D} introduction:

\begin{itemize}
  \item Arms are noticeable and pretty and they host intense star formation, H2 regions and dust.
  \item More prominent in blue, but backbone is comprised of old stars.
  \item 2/3 massive galaxies are Spirals.
  \item 3 types of spiral arm (GD, 18\%; FL, 50\%; MA, 32\%; Elmegreen et al, 2011 and Buta et al., 2015).
  \item GD have two long and well-defined spiral arms, FL have short and fragmented arm sections, MA are a fairly symmetric middle-ground, comprised of central 2-armed section which develops long ramifications in teh outer parts of the optical disc.
  \item Spiral formation is a matter of debate - including \textit{density wave theory}, \textit{tidal triggering} and \textit{swing amplification}.
  \item GD spirals and inner section of MA spirals are seen as density wave driven, whereas FL patterns are seen as swing-amplified regions of local gravitational instability
  \item Numerical models show that FL patterns are transient and reccurent, whereas GD last longer ($\sim1$ Gyr).
  \item pitch angle is a measure of arm tightness. Positive pich angle gives trailing arm, and vice versa.
  \item pitch angle is not necessarily constant, regardless of the class of the spiral arm.
  \item It is claimed that pitch angles depend on central mass concentration and atomic gas density (e.g. Kennicutt 1981; Block et al. 1994; Davis et al. 2015; Yu \& Ho 2019), galactic shear rate (e.g. Seigar et al. 2006; Grand et al. 2013), or on the steepness of the rotation curves (Seigar et al. 2005, 2014).
  \item Spiral arms are more open in galaxies with rising rotation curves (larger absolute pitch angle) and tighter in those with falling rotation curves.
  \item Very tight scaling relation reported by Davis et al. (2017) between supermassive black hole mass and the spiral pitch angle.
  \item Spirals rearrange gas and lead to the formation of disc-like bulges (e.g. Kormendy \& Kennicutt 2004). Making them important agents for the secular evolution of disc galaxies (a process in which bars also play a significant role).
  \item Spirals could be driven by bars, though this is still debated. Manifold theory suggests a strong coupling between bars and spirals (Romero-G\'omez et al. 2006, and many others). Numerical simulations show that material becomes confined to tubes (invariant manifolds) that extend from the two unstable Lagrangian points at the end of the bar ($L_1$ and $L_2$). This theory predicts a dependence of the pitch angle on the bar perturbation strength (Athanassoula et al. 2009a).
\end{itemize}

\citet{2019arXiv190910291P} introduction:

\begin{itemize}
  \item Spirals make up \textbf{one-third} (\textit{this is a typo, two-thirds is correct}) of all massive galaxies
  \item They are the site of a majority of star formation
  \item Dobbs and Baba argue that in unbarred spirals spiral arms are either transient and recurrent due to self-gravity within steller and / or gaseous discs, or are the result of tidal interactions
  \item Supported by lack of relation between pitch angle and central galaxy concentration observed in GZ (Hart 2017)
  \item Shabani et al. (2018) only find evidence of a fixed density wave in a galaxy with a strong bar, however Yu and Ho (2018, 2019) do find correlations between pitch angle and galaxy morphology
\end{itemize}

\subsection*{TL;DR}

Understanding spirals is very useful

Spiral tightness can be used as a measure of useful things

It's difficult to measure spiral tightness at scale

Bars cause predictions of tightness to change measurably
