% !TEX root = ../main.tex

\subsection{Length-weighted spiral arms}

A common method used to obtain a pitch angle of a galaxy is by identifying all

\subsection{Bayesian modelling of spiral arms in \textit{Galaxy Builder}}

We fit a number of models, including various priors on the global pitch angle distribution. We make use of PYMC3\footnote{\url{https://docs.pymc.io/}}, an open source probabilistic programming framework written in python.

\subsubsection{Spiral arms from \textit{Galaxy Builder}}

Clustered, cleaned points in polar coordinates

$$ r_\mathrm{arm} = \exp\left[\theta\tan(\phi_\mathrm{arm}) + c_\mathrm{arm}\right] + \sigma_r.$$

\subsubsection{Cot-Uniform model}

Assume the pitch angle of a galaxy's spiral arms are drawn from a Normal distribution, truncated between 0 and 90, centred on the galaxy's ``pitch angle'', $\phi_\mathrm{gal}$ with some measure of spread, $\sigma_\mathrm{gal}$, common to all galaxies:

\begin{equation}
\phi_\mathrm{arm} \sim \mathrm{TruncatedNormal}(\phi_\mathrm{gal}, \sigma_\mathrm{gal}, \mathrm{min}=0, \mathrm{max}=90).
\end{equation}

We choose a prior on $\phi_\mathrm{gal}$ such that

\begin{equation}
\cot\left(\phi_\mathrm{gal}\right) \sim \mathrm{Uniform}(\mathrm{min}=1, \mathrm{max}=4),
\end{equation}

and $\sigma_\mathrm{gal}$ of

\begin{equation}
\sigma_\mathrm{gal} \sim \mathrm{InverseGamma}(\alpha=2,\,\beta=20).
\end{equation}

We model point

\begin{equation}
c_\mathrm{arm} \sim \mathrm{Cauchy}(\alpha=0,\,\beta=10).
\end{equation}

\begin{equation}
\sigma_r \sim \mathrm{HalfCauchy}(\beta=0.2)
\end{equation}

\subsubsection{Hierarchial Normal Model}

Instead of the above model, assume

\begin{equation}
\phi_\mathrm{gal} \sim \mathrm{TruncatedNormal}(\mu_\mathrm{global}, \sigma_\mathrm{global}, \mathrm{min}=0, \mathrm{max}=90),
\end{equation}

Hyperpriors used are as follows:

\begin{equation}
\mu_\phi \sim \mathrm{Uniform}(0, 90)
\end{equation}

\begin{equation}
\sigma_\mathrm{global} \sim \mathrm{InverseGamma}(\alpha=1,\,\beta=10)
\end{equation}
