% !TEX root = ../main.tex
\subsection{Constraints on Galaxy Pitch angle}
Our hierarchical model identifies the pitch angle of individual arms ($\phi_\mathrm{arm}$) with less than {1.6\degree} of uncertainty for 95\% of arms, assuming no error on disc inclination and position angle. The pitch angle of a galaxy as a whole ($\phi_\mathrm{gal}$), however, is not well constrained. This is primarily a result of only having pitch angles measurements for a small number of arms per galaxy, and reflects the difficulty in providing a single value for the pitch angle of a galaxy containing individual arms with very different pitch angles. For galaxies with two arms identified in \textit{Galaxy Builder}, we have a mean uncertainty of ($\sigma_{\phi_\mathrm{gal}}$) of  {7.9\degree}, which decreases to {6.8\degree} and {6.0\degree} for galaxies with three and four arms respectively. This is roughly consistent with the standard error on the mean for a galaxy with $N$ arms,

\begin{equation}
  \sigma_{\phi_\mathrm{gal}} = \frac{\sigma_\mathrm{gal}}{\sqrt{N}},
\end{equation}

where $\sigma_\mathrm{gal}$ is our measure of inter-arm variability of pitch angle and has a posterior distribution of $11.0^\circ\pm 0.9^\circ$. This variability is similar to the finding of \citet{2014ApJ...790...87D} and emphaises the need for fitting algorithms to not assume all arms have the same pitch angle.

% for one-armed galaxies = {9.86\degree}

\subsection{Dependence of pitch angle on Galaxy Morphology}
\label{section:morphology_comparision}
In order to test the possible progenitor distribution of our estimated arm pitch angles, we repeatedly perform an Anderson-Darling test (\citealt{10.2307/2286009}, implemented in \textsc{Scipy}, \citealt{scipy-paper}) over each draw present in the MC trace, resulting in a distribution of Anderson-Darling statistics. We will refer to this test as the \textit{marginalized Anderson-Darling test}. We also make use of the two-sample Anderson-Darling \citep{doi:10.1080/01621459.1987.10478517} test in a similar manner.

\subsubsection{Pitch angle vs. Bulge size}

Morphological classification commonly links bulge size to spiral tightness, and such a link is implied by the Hubble Sequence (\citealt{2005ARA&A..43..581S}, \citealt{2009MNRAS.393.1531G}, \citealt{2013seg..book..155B}). Some studies have indeed reported a link between measured spiral galaxy pitch angle and bulge size (i.e. \citealt{2017MNRAS.472.2263H}, \citealt{2019ApJ...873...85D}), while others have not found any significant correlation \citep{2019MNRAS.487.1808M}. We investigate this relationship here using a measure of bulge prominence from Galaxy Zoo 2, as Equation 3 in \citet{2019MNRAS.487.1808M}:

\begin{equation}
  B_\mathrm{avg} = 0.2\times p_\mathrm{just\ noticeable} + 0.8\times p_\mathrm{obvious} + 1.0\times p_\mathrm{dominant},
\end{equation}

Where $p_\mathrm{just\ noticeable}$, $p_\mathrm{obvious}$ and $p_\mathrm{dominant}$ are the fractions of classifications indicating the galaxy's bulge was ``just noticeable'', ``obvious'' or ``dominant'' respectively.

We see no correlation between galaxy pitch angle derived from the hierarchical model and $B_\mathrm{avg}$. The Pearson correlation coefficient between the expectation value of galaxy pitch angle ($E[\phi_\mathrm{gal}]$) and $B_\mathrm{avg}$ is 0.00.

We separate our sample into galaxies with weaker bulges ($B_\mathrm{avg} < 0.28$, 83 galaxies) and those with stronger bulges ($B_\mathrm{avg} \ge 0.28$, 54 galaxies), in order to test whether their pitch angles could be drawn from significantly different distributions. A marginalized two-sample Anderson-Darling test comparing the distributions of $\phi_\mathrm{gal}$ for the samples does not find evidence that galaxy pitch angles were drawn from different distributions: we reject the null hypothesis at the 1\% level for only 1\% of the samples. Similarly comparing arm pitch angles for galaxies in the different samples results in not rejecting the null hypothesis at the 1\% level for any of the samples. The distributions of Anderson-Darling test statistic for $\phi_\mathrm{gal}$ and $\phi_\mathrm{arm}$ are shown in the upper panel of Figure \ref{fig:ad-morphology-test} in blue and orange respectively.

These results may be caused by the galaxy sample not containing many galaxies with dominant bulges: $B_\mathrm{avg}$ only varied from 0.09 to 0.75 (the allowed maximum being 1.0), with only four galaxies having $B_\mathrm{avg} > 0.5$. The split point of 0.28 was also chosen to produce evenly sized comparison samples rather than from some physical motivation. However, the lack of any form of correlation implies that there is no evidence in our data for the link between bulge size and pitch angle predicted by the Hubble sequence and observed in other studies.

\subsubsection{Pitch angle vs. Bar Strength}

One of the predictions of Manifold theory is that pitch angle increases with bar strength \citep{2009MNRAS.400.1706A}. In order to investigate this relationship in our data, we make use of Galaxy Zoo 2's bar fraction (\textit{pbar}), which is widely viewed as a good measure of bar strength (\citealt{2012MNRAS.423.1485S}, \citealt{2012MNRAS.424.2180M}) and therefore a measure of the torque applied on the disc gas.

We do not observe a correlation between \textit{pbar} and $E[\phi_\mathrm{gal}]$ (Pearson correlation coefficient of -0.05). Following \citet{2012MNRAS.424.2180M} and \citet{2012MNRAS.423.1485S}, we separate the sample into galaxies without a bar ($\mathrm{\textit{pbar}} < 0.2$), with a weak bar ($0.2 \le \mathrm{\textit{pbar}} \le 0.5$) and with a strong bar ($\mathrm{\textit{pbar}} > 0.5$). Performing marginalized three-sample Anderson-Darling tests does not find that pitch angles ($\phi_\mathrm{gal}$ or $\phi_\mathrm{arm}$) of galaxies with different bar strengths were drawn from different distributions; we do not reject the null hypothesis at the 1\% level for any samples for the test of $\phi_\mathrm{gal}$, and at the 10\% level for the test of $\phi_\mathrm{arm}$. The distributions of Anderson-Darling test statistic is shown in the lower panel of Figure \ref{fig:ad-morphology-test}.

\begin{figure*}
  \includegraphics[width=17.7cm]{plots/bulge_bar_test_results.pdf}
  \caption{The results of marginalized two-sample Anderson-Darling tests examining whether pitch angles ($\phi_\mathrm{gal}$ in blue and $\phi_\mathrm{gal}$ in orange) for galaxies with $B_\mathrm{avg} < 0.28$ and $B_\mathrm{avg} \ge 0.28$ are drawn from the same distribution (top panel), and the results of marginalized three-sample Anderson-Darling tests for galaxies with no bar ($\mathrm{\textit{pbar}} < 0.2$), a weak bar ($0.2 \le \mathrm{\textit{pbar}} \le 0.5$) and a strong bar ($\mathrm{\textit{pbar}} > 0.5$) (bottom panel). Confidence intervals are shown, with moving rightwards indicating more confidence in rejecting the null hypothesis.}
  \label{fig:ad-morphology-test}
\end{figure*}

We do not correct for variation in bulge size in this work, however predictions from Manifold theory should not be affected as bulges do not provide a strong non-axisymmetric focing. The fact that we do not find any link between bar strength and pitch angle suggests that the primary mechanism driving the evolution of the spirals in our sample is not Manifold theory. \comment{This is possibly too strong a statement?}

\subsection{Spiral Winding}
\label{section:spiral_winding}

For transient and reccurent spiral arms driven by self-gravity, \citet{2019arXiv190910291P} suggest that spiral patterns form at some maximum pitch angle ($\phi_\mathrm{max}$), continually wind up over time and finally dissapate at some minimum pitch angle ($\phi_\mathrm{min}$). They propose that, under a set of very simple assumptions, the evolution of pitch angle would be governed by

\begin{equation}
  \label{eq:winding}
  \cot{\phi} = \left[R\frac{\mathrm{d}\Omega_p}{\mathrm{d}R}\right](t - t_0) + \cot{\phi_\mathrm{max}},
\end{equation}

where $\Omega_p$ is the radially dependant pattern speed of the spiral arm and $t_0$ is the initial time at which it formed.

In QSDW theory, the pattern speed $\Omega_p$ is a constant in R, as spiral arms obey rigid-body rotation. If $\Omega_p$ instead varies with radius we would expect $\cot{\phi}$ to be uniformly distributed between $\cot{\phi_\mathrm{max}}$ and $\cot{\phi_\mathrm{min}}$.

In order to test this theory, \citet{2019arXiv190910291P} used a Kolmogorov-Smirnov test to examine the consistency of a sample of observed galaxy pitch angles with one uniform in cot. Pitch angles were measured using discrete fourier transformations in one- and two-dimensions, and as such do not account for inter-arm variations. They chose limits of $\cot{\phi} \in [1.00, 4.75]$ (roughly $11.9^\circ < \phi < 45.0^\circ$), motivated by examination of their data.

We aim to replicate their work here, using our sample and methods. We will make use of the marginalized Anderson-Darling test described above, and examine winding on a per-arm basis, as well as a per-galaxy basis. Observation of the distribution of arm pitch angles in our sample suggests limits of $15^\circ < \phi < 50.0^\circ$.

\subsubsection{Galaxy Pitch angle}

Testing the uniformity of $\cot\phi_\mathrm{gal}$ between {15\degree} and {50\degree} using a marginalized Anderson-Darling test results in rejecting the null hypothesis at the 1\% level for just 5\% of samples, with a large spread in observed test values. The full distribution of Anderson-Darling statistics can be seen in the upper panel of Figure \ref{fig:ad-cot-test}.

This result suggests that we cannot rule out a cot-uniform source distribution for galaxy pitch angle, but the large uncertainty in $\phi_\mathrm{gal}$ makes it difficult to make any conclusive statements. This result is also highly sensitive to the lower limit of $\phi$: decreasing it to {10\degree} results in us rejecting the cot-uniform model at greater than the 0.1\% level for 96\% of the posterior samples. As we have no information available on the biases present in \textit{Galaxy Builder} spiral arm classification, we choose to keep the less strict limit of {15\degree}.

\subsubsection{Arm Pitch angle}
The inconclusive result for $\phi_\mathrm{gal}$ is perhaps unsurprising: were we to assume that spiral arms are transient and reccurent instabilities, there is little reason for all of the arms to be at precisely the same evolutionary stage at the same time. This is supported by the large observed spread in inter-arm pitch angles.

If we assume instead that spirals form and wind independantly in a galaxy, and that their evolution over time can be described by Equation \ref{eq:winding}, the distribution of pitch angles of individual arms should be uniform in cot between our limits, rather than that of the galaxy's pitch angle as a whole.

Using the marginalized Anderson-Darling test we cannot reject the null hypothesis at even the 5\% level for any of the possible realizations of arm pitch angle. The resulting distribution of Anderson-Darling statistics is shown in in the lower panel of Figure \ref{fig:ad-cot-test}. This result is highly consistent with the model for spiral winding proposed by \citet{2019arXiv190910291P}, and can be seen as evidence that spirals are formed through local disc perturbations, and are primarily governed by local forces.

\begin{figure*}
  \includegraphics[width=17.7cm]{plots/combined_cot_uniform_marginalized_tests.pdf}
  \caption{The results of a marginalized Anderson-Darling test for uniformity in $\cot$ for $\phi_\mathrm{gal}$ (top panel) and $\phi_\mathrm{arm}$ (bottom panel), with values corresponding to various confidence intervals shown. Moving rightwards on the x-axis implies greater confidence in rejecting the null hypothesis.}
  \label{fig:ad-cot-test}
\end{figure*}
