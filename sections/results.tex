% !TEX root = ../main.tex
\subsection{Constraints on Galaxy Pitch angle}
Our hierarchical model identifies arm pitch angle with a hich degree of certainty (uncertainty less than 1.6\degree for 95\% of arms, assuming no error on disc inclination and position angle), however it returns a large spread of potential values for the pitch angles of galaxies, primarily caused by only measuring pitch angles of a small number of arms per galaxy. For galaxies with two arms identified in \textit{Galaxy Builder}, we have a mean uncertainty of 7.8\degree, which decreases to 6.7\degree and 5.8\degree for galaxies with three and four arms respectively.

Our measure of inter-arm variability of pitch angle, $\sigma_\mathrm{gal} = 11.02 \pm 0.95$, confirming 

\subsection{Spiral Winding}
\label{section:spiral_winding}
In order to test the possible progenitor distribution of our estimated galaxy pitch angles, we repeatedly perform an Anderson-Darling test over each draw present in the MCMC trace, resulting in a distribution of Anderson-Darling statistics. We will refer to this test as the \textit{marginalized Anderson-Darling test}. We make use of the Kolmogorov-Smirnov test in a similar manner for comparison.

We perform the marginalized Anderson-Darling test for a potential source distribution uniform in $\cot\phi$ between the limits present in \citet{2019arXiv190910291P} ($1.00 < \cot\phi < 4.75$, or roughly $11.9^\circ < \phi < 45.0^\circ$). The resulting distribution of Anderson-Darling statistics can be seen in Figure \ref{fig:ad-cot-test}. We observe that we reject the null hypothesis at the 1\% level for 69\% of the possible realizations of galaxy pitch angle. While, with our sample and methodology, we cannot unilaterally reject winding of the kind described by \citet{2019arXiv190910291P}, it suggests this model is not found in our data.

\begin{figure*}
  \includegraphics[width=17.7cm]{plots/cot_uniform_marginalized_tests.pdf}
  \caption{The results of a marginalized Anderson-Darling test (top panel), with values corresponding to various confidence intervals shown, and marginalized Kolmogorov-Smirnov test p-values (lower panel). Moving rightwards on the x-axis implies greater confidence in rejecting the null hypothesis.}
  \label{fig:ad-cot-test}
\end{figure*}

Not seeing uniformity in $\cot\phi$ is not evidence against spiral winding, nor is it evidence for static spiral structures. Our results simply suggest that a more comprehensive test is needed for observational data of this kind.


\subsection{Dependence of pitch angle on Galaxy Morphology}
\label{section:morphology_comparision}
\subsubsection{Pitch angle vs. Bulge size}
We see no correlation between galaxy pitch angle derived from the \textit{hierarchical normal model} and Galaxy Zoo 2's debiased \citep{Willett2013:1308.3496v2} \textit{pbulge}, which has been shown to be a good measure of bulge size.

We separate our sample into ``disc-dominated galaxies'' and ``obvious bulge galaxies'' using the debiased fractions from Galaxy Zoo 2 following \citet{2017MNRAS.469.3363K}, defining \textit{no bulge + just noticeable $>$ obvious + dominant} for the former and the converse for the latter. A marginalized two-sample Anderson-Darling test \citep{doi:10.1080/01621459.1987.10478517} does not find any evidence that the samples were drawn from different distributions; we do not reject the null hypothesis at the 1\% level for any samples. The distribution of Anderson-Darling test statistics is shown in the upper panel of Figure \ref{fig:ad-morphology-test}.


\subsubsection{Pitch angle vs. Bar Strength}
We see no correlation between galaxy pitch angle derived from the \textit{hierarchical normal model} and Galaxy Zoo 2's debiased \textit{pbar}, which is widely viewed as a good measure of bar strength, and therefore a measure of the torque applied on the disc gas.

Separating the sample based off of $\mathrm{\textit{pbar}} > 0.5$, and restricting to galaxies with more than 10 classifications for \textit{pbar} (as performed by \citealt{2011MNRAS.411.2026M} and \citealt{2017MNRAS.469.3363K}) and performing a marginalized two-sample Anderson-Darling test does not find that the samples were drawn from different distributions (we reject the null hypothesis at the 1\% level for only 1\% of samples). The distribution of Anderson-Darling test statistics is shown in the lower panel of Figure \ref{fig:ad-morphology-test}.

We do not account for variation in bulge size in this work, however predictions from Manifold theory should not be affected as bulges do not provide a non-axisymmetric focing.

\begin{figure*}
  \includegraphics[width=17.7cm]{plots/bulge_bar_test_results.pdf}
  \caption{The results of marginalized two-sample Anderson-Darling tests examining whether pitch angles for Bulge-dominated and Disc-dominated galaxies are drawn from the same distribution (top panel), and the results of the same test for strongly-barred vs unbarred galaxies (bottom panel).}
  \label{fig:ad-morphology-test}
\end{figure*}
