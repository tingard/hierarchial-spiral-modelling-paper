% !TEX root = ../main.tex
Spiral structure is ubiquitous in the Universe, and the pitch angle of arms in spiral galaxies provide an important observable in efforts to discriminate between different mechanisms of spiral arm formation and evolution. In this paper, we present a hierarchical Bayesian approach to galaxy pitch angle determination, using spiral arm data obtained through the \textit{Galaxy Builder} citizen science project. We present a new approach to deal with the large variations in pitch angle between different arms in a single galaxy, which obtains full posterior distributions on parameters. We make use of our pitch angles to examine previously reported links between bulge and bar strength and pitch angle, finding no correlation in our data  (with a caveat that we use observational proxies for both bulge size and bar strength which differ from other work). We test a recent model for spiral arm winding, which predicts uniformity of the cotangent of pitch angle between some unknown upper and lower limits, finding our observations are consistent with this model of transient and recurrent spiral pitch angle as long as the pitch angle at which most winding spirals dissipate or disappear is larger than 10$^\circ$.
