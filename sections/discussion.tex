% !TEX root = ../main.tex
This paper presents a new Bayesian approach to estimate galaxy pitch angle, making use of citizen science results to measure spiral arms through photometric modelling. We introduce an adaptation of the Anderson-Darling test, which we name the \textit{marginalized Anderson-Darling test}, to incorporate full Bayesian posterior probabilities and utilize this test to investigate theories governing spiral formation and evolution.

The hierarchical Bayesian approach implemented in this paper allows a more thorough examination of pitch angle than length-weighted pitch angle calculation; obtaining posterior distributions of measured parameters. It better accounts for the large varitations observed in inter-arm pitch angle than fourier analysis, which assumes all arms in a given mode have the same pitch angle. In this work we find that the mean inter-arm difference in pitch angle is $11.0^\circ\pm 0.9^\circ$.

There is no evidence in our data for the link between bulge size and pitch angle predicted by the Hubble sequence and observed in other studies.

We do not find any link between bar strength and pitch angle suggests that the primary mechanism driving the evolution of the spirals in our sample is not Manifold theory.

Our results are consistent with spiral winding of the form described by \citet{2019arXiv190910291P}, in which spiral arms are transient and reccurent, evolve through mechanisms such as swing-amplification \citep{1965MNRAS.130..125G} and which wind up over time. However, the assumptions of this model of spiral winding are highly simplistic, and it leaves many unanswered questions: what determines the limits on $\phi$? Is the spiral arm equally apparent at all pitch angles, or is a selection effect present? This result is also not evidence against QSDW, as it is possible that our distribtuion of pitch angles is dictated by other factors such as disk shear.

In this work we assume that spiral arms are equally likely to be identified and recovered at all pitch angles. This is not an unfair assumption given the amount of human effort that went into obtaining spiral arms (more so than any other pitch angle measurement method, with each galaxy receiving 30 human classifications). The galaxy sample used is not guaranteed to be representative of the general spiral population, but is comparable in size to those used in other similar studies (\citealt{2013MNRAS.436.1074S}, \citealt{2019ApJ...871..194Y}, \citealt{2019arXiv190910291P}).

The methodlogy proposed here is a robust solution to the problems facing investigation of spiral morphology, namely that of reliably identifying spiral arms, and properly accounting for the spread in pitch angles of arms within a galaxy. As with most analyses, the most impactful improvement it would be possible to make here would be to increase the cleanliness and volume of data analysed.

The processes governing the formation and evolution of spiral arms are immensely complicated, but the prevalence of spiral galaxies in the Universe, and the spiral nature of our own Milky Way, makes investigating their dynamics of fundamental importance to the scientific aims of understanding, predicting and explaining the nature of the cosmos.
