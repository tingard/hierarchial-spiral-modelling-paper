% !TEX root = ../main.tex
This paper presents a new Bayesian approach to estimate galaxy pitch angle, making use of citizen science results to measure spiral arms through photometric modelling. We introduce an adaptation of the Anderson-Darling test to incorporate full Bayesian posterior probabilities and utilize this test to investigate theories governing spiral formation and evolution.

The statistical approach implemented in this paper allows a more thorough examination of pitch angle than simpler methods of logarithmic spiral fitting, and better accounts for inter-arm pitch angle variation than fourier analysis, which assumes all arms in a given mode have the same pitch angle.

We do not find a relationship between bar strength and pitch angle, as would have been predicted by Manifold theory, and do not find evidence for the relationship between central mass concentration and pitch angle predicted by the Hubble sequence.

Our results are consistent with spiral winding of the form described by \citet{2019arXiv190910291P}; providing evidence for transient and recurrent spiral arms, the evolution of which is governed by self-gravity (such as through swing-amplification, \citealt{1965MNRAS.130..125G})) and which wind up over time. The assumptions of this model of spiral winding are also highly reductionist, and leaves many unanswered questions: what determines the limits on $\phi$? Is the spiral arm equally apparent at all pitch angles, or is a selection effect present? This result is not evidence against QSDW, as it is possible that our distribtuion of pitch angles is dictated by other factors such as disk shear and central mass concentration.

As with most analyses, the most impactful improvement it would be possible to make here would be to increase the cleanliness and volume of data analysed; due to time constraints the \textit{Galaxy Builder} galaxies used are not guaranteed to be representative. However, we believe that the methodlogy proposed here is a scalable, robust solution to the problems facing investigation of spiral morphology.
