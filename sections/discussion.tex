% !TEX root = ../main.tex
This paper presented a new Bayesian approach to estimate galaxy pitch angle, making use of citizen science results to measure spiral arms through photometric modelling. We introduce an adaptation of the Anderson-Darling test to incorporate full Bayesian posterior probabilities and utilize this test to investigate theories governing spiral formation and evolution.

The statistical approach implemented in this paper allows a more thorough approach than calculating length-weighted pitch angles, and better accounts for inter-arm pitch angle variation than fourier analysis. \comment{This is a strong statement, how can we quantify it?}.

Our method and results do not completely reject spiral winding of the form described by \citet{2019arXiv190910291P}, however this result is highly sensitive to the boundaries used; using limits of $10.0^\circ < \phi < 40.0^\circ$ results in us rejecting the null hypothesis at the 1\% level for every MCMC draw. As no physical justification was provided for the limits present in \citet{2019arXiv190910291P}, this is a far from ideal test of arm winding.

We we do not find a relationship between bar strength and pitch angle, as would have been predicted by Manifold theory, and do not find evidence for the relationship between central mass concentration and pitch angle predicted by the Hubble sequence.

This work is primarily held back by the sample used; due to time constraints the \textit{Galaxy Builder} galaxies used are not guaranteed to be representative. A larger and more complete sample is needed to be fully confident in our results, but we are confident that the methodlogy proposed here is a scalable solution to the problems facing investigation of spiral morphology.
