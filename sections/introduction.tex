% !TEX root = ../main.tex

Spiral structure is present in a majority of massive galaxies (\citealt{1989gadv.book..151B}, \citealt{2008MNRAS.389.1179L}) yet the formation mechanisms through which spiral structure originates are still hotly debated. Spirals are as diverse as the theories proposed to govern their evolution; from the quintessential pair of well-defined arcs of the grand design spiral, to the fragmented arm segments of the flocculent spiral, to the disjointed multi-armed spiral. These variatons on structure account for $18\%$, $50\%$ and $32\%$ of the population respectively (\citealt{2011ApJ...737...32E}, \citealt{2015yCat..22170032B}). The Hubble classification scheme \citep{1926ApJ....64..321H}, and its revisions and expansions (\citealt{1961hag..book.....S}, \citealt{1991rc3..book.....D}) contain detailed variations of different types of spiral galaxy, divided by the presence of a bar and ordered by how obvious spiral arm patterns are, how tightly they are wound and the prominence of a central bulge.

Arms of spiral galaxies are the source of the vast majority of star formation in the Universe, and spirals rearrange disc gas and can lead to the formation of disc-like bulges (e.g. \citealt{2004ARA&A..42..603K}). Studies of spiral morphology have found interesting correlations between spiral morphology and other galactic properties, such as a correlation between spiral tightness and central mass concentration (\citealt{2019ApJ...871..194Y}, though \citealt{2017MNRAS.472.2263H} found no such relation) and tightness and rotation curve shape (\citealt{2005MNRAS.359.1065S}, with rising rotation curves creating more open spiral structure). These predictions and observations provide compelling reasons for investigating their underlying rules and dynamics, as doing so is essential for understanding the secular evolution of disc galaxies.

Our current understanding of the mechanisms which drive spiral growth and evolution suggest that each of the different forms of spiral galaxy may be triggered primarily by different processes. Grand Design spirals are thought to have undergone a tidal interaction, be driven by a bar (as seen in gas simulations, \citealt{1976ApJ...209...53S}, \citealt{2008A&A...489..115R}, and suggested for stars by Manifold theory, \citealt{2006A&A...453...39R}, \citealt{2009MNRAS.394...67A}, \citealt{2009MNRAS.400.1706A}), or be obeying (quasi-stationary) density wave theory (QSDW theory), in which spiral arms are slowly evolving, ever-present structures in the disc \citep{1964ApJ...140..646L}. Flocculent spirals are thought to be formed through swing amplification (shearing of small gravitational instabilities in the disc), and be transient and reccurent in nature \citep{1966ApJ...146..810J}. It is recognised that methods of spiral formation are not mutually exclusive.

One of the fundamental assumptions of early work on spiral formation mechanisms (primarily QSDW) was that the disc of a galaxy, if unstable to spiral perturbations, would create a stable, static wave which would exist unchanging for many rotational periods \citep{1964ApJ...140..646L}. The motivation for static waves with small numbers of arms (with a preference for $m=2$) was primarily observational; most galaxies show spiral structure, suggesting that spirals exist for a long time or are continually rebuilt.

Many simulations demonstrate that spirals do not maintain a constant pitch angle, and instead wind-up over time due to the differential rotation of the disc \citep{2013ApJ...763...46B}. Recent research suggests that spirals arms are transient in nature, and continually dissapate and re-form \citep{2014PASA...31...35D}. These spirals can be maintained through the same mechanisms that drive QSDW spirals (i.e. WASER, \citealt{1976ApJ...205..363M}, swing amplification, \citealt{1965MNRAS.130..125G}), but do not require the idealistic disc conditions required for the formation and maintenance of a stationary wave. The pitch angles of these transient spiral arms will decrease due to the differential rotation of the disk, with the density of the arm peaking at some critical pitch angle, before dissapating to be reformed.

In this dynamic picture of spiral arms, pitch angle monotonically decreases from a spiral arm's formation to its dissapation. \citet{2019arXiv190910291P} proposes a simple test of spiral arm winding, assuming the cotangent of the pitch angle of a spiral arm evolves linearly with time. They found that the distribution of pitch angles of their sample of 86 galaxies was consistent with this prediction, evidence against QSDW theory in favour of the dynamic spirals produced in many simulations.

Spiral evolution also appears to be influenced by the presence and strength of a bar; in barred grand-design spirals the arms often appear to start from the ends of the bar. Simulations of gas in barred galaxies often demonstrate that bars can drive long-term spiral evolution \citep{2008A&A...489..115R}, or boost transient spiral structure \citep{2012MNRAS.426..167G}. Manifold theory is one attempt to determine the orbits of stars in bar-driven spiral arms: it proposes that stars in the vicinity of the unstable Lagrangian points at either end of the bar tend to escape along predictable orbits, governed by invariant manifolds. One of the primary factors influencing the shape of this invariant manifold is the relative strength of the non-axisymmetric forcing caused by the bar, with stronger bars resulting in spirals with larger pitch angles.

Many other systems contribute to spiral morphology, including potential ties to bulge fraction (\citealt{1975A&A....44..363Y}, \citealt{2013MNRAS.436.1074S}, \citealt{2019MNRAS.487.1808M}) and black hole mass (\citealt{2008ApJ...678L..93S}, \citealt{2017MNRAS.471.2187D}, \citealt{2019MS&E..571a2118A}). Stronger bulges and more more massive central black holes have both been linked to more tightly wound spiral arms.

This paper makes use of the classification data and fitted photometric models obtained through the \textit{Galaxy Builder} citizen science project for the 196 spiral galaxies present in \Lingard. In this paper we focus on the use of measured spiral tightness (quantified using pitch angle, \citealt{1987gady.book.....B}) as a probe into the dynamical mechanisms governing a spiral galaxy's evolution. We make use of Bayesian hierarchical modelling to measure galaxy pitch angle from the spiral arm clusters producted by \textit{Galaxy Builder}.

Section \ref{section:spiral_winding} investigates spiral arm winding using the test derived by \cite{2019arXiv190910291P} (uniformity of galaxy pitch angle in $\cot\;\phi$) and concludes using a marginalized Anderson-Darling test that we cannot unilaterally reject winding of this form at the 1\% level. Section \ref{section:morphology_comparision} examines the correlation between pitch angle and bulge size implied by the Hubble sequence, and pitch angle and bar strength implied by Manifold theory, and find no significant correlation.
