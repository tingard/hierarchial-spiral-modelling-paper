% !TEX root = ../main.tex


% BEGIN HERE
Spiral structure is present in a majority of massive galaxies \citep{2008MNRAS.389.1179L}, yet the formation mechanisms through which spiral structure originates are still hotly debated. Spirals themselves are as diverse as the theories proposed to govern their evolution; ranging from the quintessential pair of well-defined arcs of the grand design spiral, to the patchy and fragmented arm segments of the flocculent spiral, to the disjointed multi-armed spiral. These variatons on structure account for $18\%$, $50\%$ and $32\%$ of the population respectively (\citealt{2011ApJ...737...32E}, \citealt{2015yCat..22170032B}). Our current understanding of the mechanisms which drive spiral growth and evolution suggest that each of the different forms of spiral galaxy may be triggered primarily by different processes. Grand Design spirals are thought to have undergone a tidal interaction, be driven by a bar (as suggested by Manifold theory, \citealt{2011hsa6.conf..548R}), or be obeying quasi-stationary density wave theory, in which spiral arms are slowly evolving, ever-present structures in the disc \citep{1964ApJ...140..646L}. Flocculent spirals are thought to be formed through swing amplification (shearing of small gravitational instabilities in the disc), and be transient and reccurent in nature \citep{1966ApJ...146..810J}. However it is recognised that no two methods of spiral formation are mutually exclusive.

Arms of spiral galaxies are the source of the vast majority of star formation in the Universe, and spirals rearrange disc gas and can lead to the formation of disc-like bulges (e.g. \citealt{2004ARA&A..42..603K}). Studies of spiral morphology have found interesting correlations between spiral morphology and other galactic properties, such as a correlation between spiral tightness and central mass concentration (\citealt{2019ApJ...871..194Y}, \citealt{2015PhDT........14D}, though \citealt{2017MNRAS.472.2263H} found no such relation) and tightness and rotation curve shape (\citealt{2005MNRAS.359.1065S}, with rising rotation curves creating more open spiral structure). These predictions and observations provide compelling reasons for investigating their underlying rules and dynamics, as doing so is essential for understanding the secular evolution of disc galaxies.

Many methodologies have been proposed and implemented to measure spiral arm properties, including visual inspection (i.e. \citealt{2015A&A...582A..86H}), fourier analysis (i.e. \citealt{2019arXiv190804246D}), more complex automated identification (i.e. SpArcFiRe, \citealt{2014ApJ...790...87D}), and combinations of automated methods and human classifiers \citep{2017MNRAS.472.2263H}. One potentially underused method of obtaining measurements of spirals is through photometric fitting of spiral structure, as possible using tools such as \textsc{GALFIT} \citep{2010AJ....139.2097P} and \textit{Galaxy Builder} \comment{(Lingard et al, in prep)}. These methods attempt to localize light from an image of a galaxy into distinct subcomponents, such as a galaxy disc, bulge, bar and spiral arms, generally finding the optimum solution using computational optimization.

This paper makes use of the photometric models obtained through the \textit{Galaxy Builder} citizen science project \comment{(Lingard et al, in prep)} which measure spiral brightness, thickness and tightness. In this paper we focus on the use of measured spiral tightness (quantified using pitch angle, \citealt{1987gady.book.....B}) as a probe into the dynamical mechanisms governing a spiral galaxy's evolution. We make use of multilevel modelling to test for signs of spiral arm winding using the predictions derived by \citealt{2019arXiv190910291P} (uniformity of galaxy pitch angle in $\cot\;\phi$) and conclude that our sample of galaxies do not show winding of this form. We model galaxy pitch angle as a truncated normal distribution make use of the resulting pitch angles to examine the correlation between pitch angle and bulge size, implied by the Hubble sequence, and bar strength, implied by Manifold theory.
