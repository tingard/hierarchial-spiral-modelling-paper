% !TEX root = ../main.tex


% BEGIN HERE
Spiral structure is present in a majority of massive galaxies (\citealt{1989gadv.book..151B}, \citealt{2008MNRAS.389.1179L}), yet the formation mechanisms through which spiral structure originates are still hotly debated. Spirals themselves are as diverse as the theories proposed to govern their evolution; ranging from the quintessential pair of well-defined arcs of the grand design spiral, to the patchy and fragmented arm segments of the flocculent spiral, to the disjointed multi-armed spiral. These variatons on structure account for $18\%$, $50\%$ and $32\%$ of the population respectively (\citealt{2011ApJ...737...32E}, \citealt{2015yCat..22170032B}). Our current understanding of the mechanisms which drive spiral growth and evolution suggest that each of the different forms of spiral galaxy may be triggered primarily by different processes. Grand Design spirals are thought to have undergone a tidal interaction, be driven by a bar (as suggested by Manifold theory, \citealt{2011hsa6.conf..548R}), or be obeying quasi-stationary density wave theory, in which spiral arms are slowly evolving, ever-present structures in the disc \citep{1964ApJ...140..646L}. Flocculent spirals are thought to be formed through swing amplification (shearing of small gravitational instabilities in the disc), and be transient and reccurent in nature \citep{1966ApJ...146..810J}. However it is recognised that no two methods of spiral formation are mutually exclusive.

Arms of spiral galaxies are the source of the vast majority of star formation in the Universe, and spirals rearrange disc gas and can lead to the formation of disc-like bulges (e.g. \citealt{2004ARA&A..42..603K}). Studies of spiral morphology have found interesting correlations between spiral morphology and other galactic properties, such as a correlation between spiral tightness and central mass concentration (\citealt{2019ApJ...871..194Y}, \citealt{2015PhDT........14D}, though \citealt{2017MNRAS.472.2263H} found no such relation) and tightness and rotation curve shape (\citealt{2005MNRAS.359.1065S}, with rising rotation curves creating more open spiral structure). These predictions and observations provide compelling reasons for investigating their underlying rules and dynamics, as doing so is essential for understanding the secular evolution of disc galaxies.

\subsection{Measuring galaxy pitch angle}

Many methodologies have been proposed and implemented to measure spiral arm properties, including visual inspection (\citealt{2015A&A...582A..86H}), fourier analysis (i.e. \textsc{2DFFT}, \citealt{2012ApJS..199...33D}), texture analysis (i.e. SpArcFiRe, \citealt{2014ApJ...790...87D}), and combinations of automated methods and human classifiers \citep{2017MNRAS.472.2263H}. One potentially underused method of obtaining measurements of spirals is through photometric fitting of spiral structure, as possible using tools such as \textsc{GALFIT} \citep{2010AJ....139.2097P} and \textit{Galaxy Builder} (\comment{Lingard et al, in prep}). These methods attempt to localize light from an image of a galaxy into distinct subcomponents, such as a galaxy disc, bulge, bar and spiral arms, generally finding the optimum solution using computational optimization. This optimization process, however, is often not robust for complex, many-component models and requires significant supervision to converge to a physically meaningful result \citep{Gao2017:1709.00746v1}. \comment{Lingard et al, in prep} successfully solve this problem through the use of citizen science to provide the starting point for a computational fit.

A common assumption when measuring galaxy pitch angle is that observed spiral arms have a constant pitch angle, these spirals are known as logarithmic spirals and are described by

\begin{equation}
r = A\,e^{\theta\tan\phi}.
\end{equation}

One method used to obtain a pitch angle of a galaxy from identified arm segments is by taking the weighted mean of their pitch angles (as used by \textsc{SpArcFiRe}, \citealt{2014ApJ...790...87D}), where weighting is determined by the length of the arc segment, with longer arcs having higher weights, i.e. for a galaxy where we have identified $N$ arm segments, each with length $L_i$ and pitch angle $\phi_i$

\begin{equation}
  \phi_\mathrm{gal} = \left(\sum_{i=1}^{N}L_i\right)^{-1}\sum_{i=1}^{N}L_i \phi_i.
\end{equation}

The most commonly used measurement of uncertainty of length-weighted pitch angles is the \comment{unweighted?} sample variance between the arm segments identified.

One notable drawback of the use of length-weighted arms is the sensitivity of the end result to the number and quality of the spiral arm segments identified \comment{...}


This paper makes use of the photometric models obtained through the \textit{Galaxy Builder} citizen science project for the 196 spiral galaxies present in \comment{Lingard et al, in prep}. In this paper we focus on the use of measured spiral tightness (quantified using pitch angle, \citealt{1987gady.book.....B}) as a probe into the dynamical mechanisms governing a spiral galaxy's evolution. We make use of Bayesian hierarchical modelling to measure galaxy pitch angle from the spiral arm clusters producted by \textit{Galaxy Builder}. We test for signs of spiral arm winding using the predictions derived by \citealt{2019arXiv190910291P} (uniformity of galaxy pitch angle in $\cot\;\phi$) and conclude using a marginalized Anderson-Darling test that we \comment{cannot unilaterally reject winding of this form at the 1\% level}. Section \ref{section:morphology_comparision} examines the correlation between pitch angle and bulge size, implied by the Hubble sequence, and bar strength, implied by Manifold theory, and find \comment{no significant correlation}.
