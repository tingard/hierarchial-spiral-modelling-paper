% !TEX root = ../main.tex
\label{section:summary}
This paper presents a new Bayesian approach to estimate galaxy pitch angle, making use of citizen science results to measure spiral arms through photometric modelling. We introduce an adaptation of the Anderson-Darling test, which we name the \textit{marginalized Anderson-Darling test}, to incorporate full Bayesian posterior probabilities and use this test to investigate theories governing spiral formation and evolution.

The hierarchical Bayesian approach implemented in this paper allows a more thorough examination of pitch angle than length-weighted pitch angle calculation obtaining posterior distributions of measured parameters. It better accounts for the large variations observed in inter-arm pitch angle than Fourier analysis, which assumes all arms in a given symmetric mode have the same pitch angle. In this work, we find that the mean inter-arm difference in pitch angle is $11.0^\circ\pm 0.9^\circ$.

There is no evidence in our data for the link between bulge size and pitch angle predicted by the Hubble sequence and observed in other studies (see Section \ref{section:morphology-comparison-bulge}).

We do not find any link between {\bf our measure of} bar strength and pitch angle in our sample. However, rather that a direct measure of bar strength, we make use of an available parameter which correlates with bar strength, so at best this observation {\bf is suggestive} that the primary mechanism driving the evolution of the spirals in our sample {\bf may not be} Manifold theory (see Section \ref{section:morphology-comparison-bar}). \textbf{Since this is not the measure of bar strength predicted to correlate with pitch angle by \citet{2009MNRAS.400.1706A}, and those authors caution that the details of the bar strength measure can wash out the predicted correlation, this is not strong evidence against Manifold theory models}. 

Our results are consistent with spiral winding of the form described by \citet{2019arXiv190910291P}, in which spiral arms are transient and recurrent, evolve through mechanisms such as swing-amplification \citep{1965MNRAS.130..125G} and which wind up over time. This model predict a distribution of pitch angles that is uniform in cotangent space across some range. No prediction is provided as to what that range should be. Our data are consistent with this model, if the minimum pitch angle is $\phi_\mathrm{min} = $ {15\degree}, but rule it out if the minimum pitch angle is $\phi_\mathrm{min} = $ {10\degree}. The assumptions of this model of spiral winding are highly simplistic, and it leaves many unanswered questions: what determines the limits on $\phi$? Is the spiral arm equally apparent at all pitch angles, or is a selection effect present? Our observations suggest that any further development of this model needs to predict that the minimum pitch angle, $\phi_\mathrm{min}>${10\degree}.This result is also not evidence against QSDW, as our distribution of pitch angles may be dictated by other factors such as disc shear.

In this work, we assume that spiral arms are equally likely to be identified and recovered at all pitch angles, which suggests the absence of galaxies at low pitch angles is not {\bf due to an inability of us to measure such arms}. This is not an unfair assumption given the amount of human effort that went into obtaining spiral arm measurements (more so than any other pitch angle measurement method, with each galaxy receiving at least 30 human classifications). The galaxy sample used is {\bf a random subset of a volume limited sample (see Figure \ref{fig:stellarmass}), and} is comparable in size to those used in other similar studies (\citealt{2013MNRAS.436.1074S}, \citealt{2019ApJ...871..194Y}, \citealt{2019arXiv190910291P}). The sample covers a range of masses ($9.45 < \log(M_* / M_\odot) < 11.05$) and spiral types, {\bf however it is possible that tightly wound spirals are preferentially missed by the pre-selection from {\it Galaxy Zoo}, if they are less obviously identified as having spirals at these distances ($0.02<z<0.055$). The \citet{1981AJ.....86.1847K} sample of 113 much more nearby spirals (all at $z<0.019$ and most at $z<0.009$), includes several with arms at much lower pitch angles, but very few arms which are loosely wound (none with $\cot \phi > 31^\circ$), meaning it does not match the same $\cot \phi$ constant model well, although as at the time the sample was set by data availability, it is unclear how conclusive this is. In a future version of {\it Galaxy Builder} we intend to include this sample as a comparison set.}

We have presented evidence that the methodology proposed here is a robust solution to the problems facing investigation of spiral morphology, namely that of reliably identifying spiral arms, and properly accounting for the spread in pitch angles of arms within a galaxy. This is one of the largest samples for which this test has been done and is scaleable to larger samples; such a sample would make possible further comparisons, such as splitting galaxies into spiral type (grand design / many-armed / flocculent), examining the differences between populations, investigating if the interarm spread depends on other galaxy properties.

The processes governing the formation and evolution of spiral arms are complicated, but the prevalence of spiral galaxies in the Universe, their impact for understanding star formation, and the spiral nature of our own Milky Way, makes investigating their dynamics of fundamental importance to the scientific aims of understanding, predicting and explaining the nature of the cosmos.
