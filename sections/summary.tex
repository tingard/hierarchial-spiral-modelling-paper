% !TEX root = ../main.tex
\label{section:summary}
This paper presents a new Bayesian approach to estimate galaxy pitch angle, making use of citizen science results to measure spiral arms through photometric modelling. We introduce an adaptation of the Anderson-Darling test, which we name the \textit{marginalized Anderson-Darling test}, to incorporate full Bayesian posterior probabilities and utilize this test to investigate theories governing spiral formation and evolution.

The hierarchical Bayesian approach implemented in this paper allows a more thorough examination of pitch angle than length-weighted pitch angle calculation; obtaining posterior distributions of measured parameters. It better accounts for the large variations observed in inter-arm pitch angle than Fourier analysis, which assumes all arms in a given \textbf{symmetric} mode have the same pitch angle. In this work, we find that the mean inter-arm difference in pitch angle is $11.0^\circ\pm 0.9^\circ$.

There is no evidence in our data for the link between bulge size and pitch angle predicted by the Hubble sequence and observed in other studies (see Section \ref{section:morphology-comparison-bulge}).

We do not find any link between bar strength and pitch angle \textbf{in our sample. We make use of a novel (not tested) measure of bar strength, so at best this observation} suggests that the primary mechanism driving the evolution of the spirals in our sample is not Manifold theory (see Section \ref{section:morphology-comparison-bar}), \textbf{but cannot completely rule it out.}

Our results are consistent with spiral winding of the form described by \citet{2019arXiv190910291P}, in which spiral arms are transient and recurrent, evolve through mechanisms such as swing-amplification \citep{1965MNRAS.130..125G} and which wind up over time, \textbf{as long as the minimum pitch angle before the majority of arms dissipate or disappear is $\phi_\mathrm{min} = $ {15\degree}}. The assumptions of this model of spiral winding are highly simplistic, and it leaves many unanswered questions: what determines the limits on $\phi$? Is the spiral arm equally apparent at all pitch angles, or is a selection effect present? \textbf{Our observations suggest that any further development of this model needs to predict that the minimum pitch angle, $\phi_\mathrm{min}>${10\degree}.} This result is also not evidence against QSDW, as our distribution of pitch angles may be dictated by other factors such as disk shear.

In this work, we assume that spiral arms are equally likely to be identified and recovered at all pitch angles, \textbf{which suggests the absence of galaxies at low pitch angles is not a selection effect}. This is not an unfair assumption given the amount of human effort that went into obtaining spiral arm measurements (more so than any other pitch angle measurement method, with each galaxy receiving at least 30 human classifications). The galaxy sample used is not selected to be representative of the general spiral population, but is comparable in size to those used in other similar studies (\citealt{2013MNRAS.436.1074S}, \citealt{2019ApJ...871..194Y}, \citealt{2019arXiv190910291P}) and covers a range of masses ($9.45 < \log(M_* / M_\odot) < 11.05$ and spiral types.

\textbf{We have presented evidence that} the methodology proposed here is a robust solution to the problems facing investigation of spiral morphology, namely that of reliably identifying spiral arms, and properly accounting for the spread in pitch angles of arms within a galaxy. This is one of the largest samples for which this test has been done \textbf{and is scaleable to larger samples}; such a sample would make possible further comparisons, such as splitting galaxies into spiral type (grand design / many-armed / flocculent), examining the differences between populations, \textbf{investigating if the interarm spread depends on other galaxy properties.}

The processes governing the formation and evolution of spiral arms are complicated, but the prevalence of spiral galaxies in the Universe, their impact for understanding star formation, and the spiral nature of our own Milky Way, makes investigating their dynamics of fundamental importance to the scientific aims of understanding, predicting and explaining the nature of the cosmos. 
